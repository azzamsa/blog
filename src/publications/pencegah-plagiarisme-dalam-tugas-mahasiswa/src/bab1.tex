%%%%%%%%%%%%%%%%%%%%%%%%%%%%%%%%%%%%%%%%%%%%%%%%%%%%%%%%%%%%%%%%%%%%%%%
% BAB 1
%%%%%%%%%%%%%%%%%%%%%%%%%%%%%%%%%%%%%%%%%%%%%%%%%%%%%%%%%%%%%%%%%%%%%%%

\mychapter{1}{BAB 1 PENDAHULUAN}

\section{Latar Belakang}

% plagiarisme itu apa (konteks)
Plagiarisme dalam lingkungan akademisi adalah penggunaan kata atau ide dari
sebuah sumber tanpa menyertakan pengakuan sebagaimana ditentukan oleh
prinsip-prinsip akademik~\parencite{meuschke2013state}.
\textcite{meuschke2013state} berpendapat bahwa plagiarisme dalam lingkungan
akademisi tidak selalu dapat dikaitkan dengan pencurian. Plagiarisme tersebut
dapat terjadi karena unsur tidak sengaja, seperti penulis yang lupa memasukkan
sitiran, melakukan sitiran pada sumber yang salah, atau tidak sengaja melakukan
sitiran pada dirinya sendiri. Terdapat beberapa tipe plagiarisme, diantaranya
menyalin kata, parafrase, dan menerjemahkan kata \parencite{kiss2013loopholes}.

% 1. Objek penelitian [A. mengapa plagiarisme]
Plagiarisme menjadi hal yang seakan tak terpisahkan di lingkungan akademisi.
Mahasiswa seringkali melakukan plagiarisme dalam tugas maupun
ujian. Ketersediaan sumber daya \emph{internet} dan kemudahan mencari informasi
berkontribusi pada munculnya plagiarisme oleh siswa \parencite{born2003teaching}.
Plagiarisme di Indonesia tidak hanya dilakukan oleh mahasiswa saja, melainkan
juga peneliti dan dosen dari universitas-universitas ternama
\parencite{agustina2017exploring}. \textcite{kiss2013loopholes} juga
menyampaikan bahwa banyak insiden plagiarisme yang menyebabkan karier
tokoh-tokoh besar hancur. Banyaknya motivasi untuk melakukan plagiarisme
di tingkat universitas, seperti mahasiswa yang bekerja paruh waktu dan
dilema sosial, membuat tingkat plagiarisme di universitas semakin tinggi
\parencite{park2004rebels}. Oleh karena alasan-alasan tersebut diperlukan
penanganan plagiarisme.

% 2. Metode-metode yang ada
Metode dalam plagiarisme dapat dikategorikan menjadi dua. Pertama
adalah metode pendeteksian atau \emph{plagiarism detection}. Metode
ini menitikberatkan pada proses pendeteksian apakah seorang siswa
melakukan plagiarisme setelah siswa tersebut menyelesaikan aktivitas
yang memungkinkan terjadinya plagiarisme, seperti kuis tugas rumah
(\emph{take home quiz}), ujian harian dan ujian akhir. Hasil dari
tugas atau ujian tersebut akan melewati proses pendeteksian plagiarisme,
seperti menggunakan alat bantu perangkat lunak \emph{plagiarism
  detection software} \parencite{neill2004web} atau dilakukan secara
manual. Metode kedua adalah metode pencegahan atau \emph{plagiarism
  prevention}. Penanganan plagiarisme pada metode pencegahan
ditekankan pada saat sebelum siswa memulai aktivitas atau selama
mengerjakan aktivitas yang memungkinkan terjadinya plagiarisme. Proses
pencegahannya, seperti mengimplementasikan \emph{formative assessment}
pada siswa \parencite{leung2017instructional}, mengedukasi siswa
tentang plagiarisme, mengajarkan cara sitiran dengan benar, membangun
atmosfer yang sehat di kelas \parencite{mclafferty2004electronic}, dan
melakukan pencegahan dengan bantuan perangkat lunak \parencite{hellas2017plagiarism}.

% 3. Kelebihan dan kekurangan metode
% Kelemahan PDS
Penggunaan \emph{Plagiarism Detection Software} atau \emph{PDS}
memudahkan para guru untuk menemukan siswa tersangka
plagiarisme. Tetapi penggunaan \emph{PDS} memiliki kekurangan, seperti
dapat menyebabkan paradigma siswa kepada guru bagaikan hubungan
polisi-kriminal daripada guru-murid \parencite{howard2002don},
membentuk pola pikir siswa menjadi nilai \emph{oriented} dari pada
\emph{goal oriented} \parencite{dweck1988social}, membuat siswa
mencari cara-cara baru agar tidak terdeteksi \emph{PDS}
\parencite{kiss2013loopholes}; mengubah teks menjadi gambar,
membubuhkan \emph{invinsible character} dan mengubah \emph{character
  map}, yaitu mengubah huruf ``a'' menjadi karakter ``c'' dan mengubah
huruf lain ke karakter lain; lemah dalam menganalisis dokumen yang
sudah diterjemahkan \parencite{meuschke2013state}, memiliki batasan
dalam proses menganalisis berbagai macam gaya sitiran
\parencite{modiba2016evaluating} dan berbagai macam bahasa
\parencite{martins2014plagiarism}, tidak dapat melacak bantuan
eksternal yang datang dari luar kelas
\parencite{hellas2017plagiarism}, harga yang ditawarkan tidak
terjangkau sehingga negara berkembang tidak dapat mengadopsinya
\parencite{agustina2017exploring, modiba2016evaluating} dan
pelanggaran hak-hak intelektual dan hak milik siswa seperti yang
dilakukan \emph{Turnitin}. \emph{Turnitin} menyalin dan menyimpan
seluruh tugas siswa ke dalam basis datanya, terlepas siswa tersebut
melakukan plagiarisme atau tidak. Oleh karena itu beberapa universitas
meninggalkannya \parencite{park2004rebels}.

% Plagiarism detection sudah lemah, apa solusinya ?
% mengapa plagiarism prevention
\emph{PDS} tidak dapat mendeteksi semua kasus yang ada pada
permasalahan plagiarisme \parencite{martins2014plagiarism}. Oleh karena
itu, pencegahan plagiarisme sangat diutamakan
\parencite{garner2012stop}. Metode pencegahan plagiarisme
(\emph{plagiarism prevention}) dengan bantuan perangkat lunak
memberikan kemampuan pada guru untuk menganalisis kecurangan selama
proses pengerjaan tugas, seperti mendeteksi plagiarisme dari sumber
non-digital dan bantuan eksternal. Selain itu, guru dapat menganalisis
pola proses pengerjaan dari \emph{log} yang dihasilkan sehingga dapat
menganalisis adanya kejanggalan maupun menemukan siswa yang lemah pada
pelajaran tertentu. Kemampuan-kemampuan tersebut tidak dapat dilakukan oleh
\emph{PDS} \parencite{hellas2017plagiarism}.

% 4. masalah pada metode yang dipilih
\textcite{hellas2017plagiarism} menggunakan metode pencegahan
plagiarisme (\emph{plagiarism prevention}) dengan bantuan perangkat
lunak. Metode tersebut merekam waktu mulai, waktu selesai dan alamat
\emph{IP} siswa dengan bantuan perangkat lunak \emph{JPlag}. Pada
akhir prosesnya dilakukan analisis hasil tugas menggunakan algoritme
\emph{edit-distance}. Rekaman waktu mulai dan waktu selesai digunakan
untuk menemukan siswa yang mengerjakan pada waktu yang bersamaan,
rekaman alamat \emph{IP} digunakan untuk menenemukan siswa yang
mengerjakan pada tempat yang sama, dan analisis hasil akhir tugas
menggunakan \emph{edit-distance} digunakan untuk menemukan siswa yang
mendapatkan bantuan dari sumber eksternal atau tidak dikerjakan dengan
kemampuannya sendiri. Tetapi metode ini memiliki beberapa kelemahan,
seperti tidak merekam segala aktivitas yang dilakukan selama
mengerjakan tugas sehingga guru hanya menganalisis hasil akhir tugas
dan mengabaikan proses pengerjaan tugas. Hal ini dapat dicurangi siswa
dengan menyelesaikan tugas dengan membeli
\parencite{leung2017instructional}. Analisis bantuan eksternal hanya
dilakukan pada hasil akhir tugas sehingga kecurangan selama
mengerjakan tugas dapat dicurangi dengan bantuan mencari
jawaban melalui \emph{browser} maupun berkolaborasi menggunakan sosial
media. Metode ini juga tidak memiliki \emph{log} pengerjaan sehingga
guru tidak dapat menilai usaha yang telah dilakukan siswa
\parencite{hellas2017plagiarism}. Selain itu, proses rekaman
dilakukan dengan bantuan perangkat lunak yang terikat dengan
\emph{platform} tertentu sehingga memiliki batasan bentuk tugas yang
akan diberikan dan batasan pilihan kakas bantu yang akan digunakan
siswa.

% 5. solusi perbaikan metode
Kelemahan yang ada pada metode yang dilakukan
\textcite{hellas2017plagiarism} akan disempurnakan dengan
menyelesaikan masalah yang telah disebutkan sebelumnya dengan cara
menambahkan proses \emph{snapshotting} dan \emph{user activity
  logging} selama pengerjaan tugas. Proses \emph{snapshotting} akan
merekam seluruh perubahan dokumen. Oleh karena
itu, \emph{snapshotting} dapat menghindari adanya siswa yang melakukan
kecurangan melalui bantuan eksternal seperti membeli tugas dari orang
lain. \emph{User activity logging} merekam seluruh aktivitas
siswa selama pengerjaan tugas, seperti nama \emph{login}, nama
peranti yang digunakan, \emph{window} yang sedang aktif, dan seluruh
\emph{window} yang terbuka. Maka \emph{user activity logging} dapat
menghindari adanya bantuan eksternal melalui \emph{browser} maupun perangkat
lunak sosial media ketika proses mengerjakan tugas.
Proses perekaman data menggunakan perangkat lunak yang bersifat
\emph{agnostic} sehingga guru tidak memiliki batasan dalam memilih
bentuk tugas yang akan diberikan dan siswa memiliki kebebasan
menggunakan kakas bantu yang disukai. Selain itu, seluruh proses
pengerjaan tugas direkam sehingga guru dapat menganalisis seberapa
besar kesugguhan dan seberapa banyak keterlibatan siswa dalam tugas
yang diberikan.

% 6. rangkuman tujuan penelitian
Penelitian ini akan mengimplementasikan metode \emph{snapshotting} dan
\emph{user activity logging}, dengan tambahan algoritme
\emph{edit-distance} yang didapatkan dari penelitian sebelumnya. Metode
\emph{snapshotting} akan dibangun di atas \emph{distributed version
control} untuk merekam seluruh perubahan dokumen tugas. Metode \emph{activity
logging} digunakan untuk merekam segala aktivitas siswa selama
pengerjaan tugas. Algoritme \emph{edit-distance} digunakan untuk menghitung nilai
\emph{edit-distance} pada berkas tugas. Maka seluruh usaha, keputusan, keterlibatan dan
aktivitas siswa dapat dianalisis oleh guru.

\section{Rumusan Masalah}

Berdasarkan permasalahan yang diangkat, maka dapat dirumuskan masalah
sebagai berikut:

\begin{enumerate}
\item Bagaimana hasil analisis kebutuhan dalam pengembangan aplikasi pencegah
  plagiarisme dengan menggunakan metode \emph{snapshotting} dan
  \emph{user activity logging}?
\item Bagaimana hasil perancangan dan implementasi aplikasi pencegah
  plagiarisme dengan menggunakan metode \emph{snapshotting} dan
  \emph{user activity logging}?
\item Bagaimana hasil pengujian dari pengembangan aplikasi pencegah
  plagiarisme dengan menggunakan metode \emph{snapshotting} dan
  \emph{user activity logging}?
\end{enumerate}

\section{Tujuan}

Tujuan penelitian ini adalah sebagai berikut:

\begin{enumerate}
\item Menganalisis kebutuhan yang diperlukan untuk membangun aplikasi
  pencegah plagiarisme dengan menggunakan metode \emph{snapshotting}
  dan \emph{user activity logging}.
\item Merancang dan mengimplementasikan aplikasi pencegah plagiarisme
  dengan menggunakan metode \emph{snapshotting} dan \emph{user
    activity logging} sesuai dengan hasil analisis kebutuhan.
\item Menguji aplikasi pencegah plagiarisme dengan menggunakan metode
  \emph{snapshotting} dan \emph{user activity logging} untuk
  memastikan bahwa aplikasi yang dibangun sesuai dengan hasil
  rancangan dan kebutuhan sebelumnya.
\end{enumerate}


\section{Manfaat}

Manfaat yang didapatkan dari penelitian ini adalah:

\begin{enumerate}
\item Menyediakan aplikasi yang dapat mencegah siswa melakukan
  plagiarisme.
\item Menyediakan aplikasi yang dapat memberikan petunjuk siswa yang
  melakukan kecurangan selama mengerjakan tugas.
\item Menyediakan aplikasi yang dapat memberikan petunjuk siswa yang
  lemah pada suatu pelajaran tertentu.
\end{enumerate}

\section{Batasan Masalah}

Penelitian ini memiliki batasan-batasan sebagai berikut:

\begin{enumerate}
\item Sistem yang dikembangkan hanya dapat mengolah tugas dengan bentuk \emph{plain text
    format}, sistem tidak dapat mengolah \emph{binary format}.
\item Sistem yang dikembangkan hanya dapat berjalan pada sistem operasi \emph{GNU/Linux}.
\item Sistem yang dikembangkan membutuhkan perangkat lunak pengolah kata yang memiliki
  fitur \emph{autosave}.
\end{enumerate}


\section{Sistematika Pembahasan}

Penjelasan singkat mengenai pembahasan dan langkah-langkah yang
dilakukan dalam setiap bab pada penelitian ini tersusun sebagai
berikut:\\

\noindent
\textbf{BAB I:\@ PENDAHULUAN}

Bab ini menjelaskan tentang latar belakang pengembangan aplikasi
pencegah plagiarisme dengan menggunakan metode \emph{snapshotting} dan
\emph{user activity logging}, rumusan masalah,
tujuan penelitian, manfaat penelitian, batasan masalah, dan
sistematika pembahasan.\\

\noindent
\textbf{BAB II:\@ LANDASAN KEPUSTAKAAN}

Bab ini menjelaskan tentang penelitian-penelitian sebelumnya yang
terkait dengan pencegahan plagiarisme, teori-teori yang menjadi
landasan dalam penelitian, dan penjelasan
kakas bantu yang digunakan dalam penelitian.\\
\newpage % manual-adj
\noindent
\textbf{BAB III:\@ METODOLOGI PENELITIAN}

Bab ini menjelaskan tahapan-tahapan penelitian yang dilakukan untuk
mengembangkan aplikasi pencegah plagiarisme dengan menggunakan metode
\emph{snapshotting} dan \emph{user activity logging}. Tahapan-tahapan
tersebut meliputi studi literatur, rekayasa kebutuhan,
perancangan, implementasi, pengujian, dan penarikan kesimpulan dan
saran.\\

\noindent
\textbf{BAB IV:\@ REKAYASA KEBUTUHAN}

Bab ini menjelaskan proses rekayasa kebutuhan untuk membangun
sistem pencegah plagiarisme dengan menggunakan metode
\emph{snapshotting} dan \emph{user activity logging}. Tahapan
rekayasa kebutuhan meliputi analisis kebutuhan, identifikasi
aktor, spesifikasi dan manajemen kebutuhan, dan pemodelan
kebutuhan. Kebutuhan tersebut meliputi kebutuhan fungsional dan
non-fungsional. Pada tahapan ini juga dijelaskan deskripsi umum sistem. \\

\noindent
\textbf{BAB V:\@ PERANCANGAN \& IMPLEMENTASI}

Bab ini menjelaskan proses perancangan dan implementasi dari
aplikasi yang dibangun. Tahapan perancangan meliputi perancangan
arsitektur, komponen, dan antarmuka. Tahapan implementasi meliputi
implementasi kode, dan antarmuka.\\

\noindent
\textbf{BAB VI:\@ PENGUJIAN}

Bab ini menjelaskan proses pengujian perangkat lunak yang
dibangun. Pengujian yang dilakukan berupa pengujian unit, pengujian
integrasi, pengujian validasi, \emph{automated testing}, dan pengujian
\emph{compatibility}. \\

\noindent
\textbf{BAB VII:\@ PENUTUP}

Bab penutup menjelaskan kesimpulan dari hasil penelitian yang telah
dikerjakan, dan saran-saran untuk perbaikan penelitian selanjutnya.\\

%%% Local Variables:
%%% coding: utf-8
%%% mode: latex
%%% TeX-engine: xetex
%%% TeX-master: "skripsi"
%%% ispell-local-dictionary: "id"
%%% End:
