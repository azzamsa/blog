%%%%%%%%%%%%%%%%%%%%%%%%%%%%%%%%%%%%%%%%%%%%%%%%%%%%%%%%%%%%%%%%%%%%%%%
% BAB 7
%%%%%%%%%%%%%%%%%%%%%%%%%%%%%%%%%%%%%%%%%%%%%%%%%%%%%%%%%%%%%%%%%%%%%%%

\mychapter{7}{BAB 7 PENUTUP}

\section{Kesimpulan}

Kesimpulan dari hasil penelitian yang dilakukan adalah sebagai berikut:

\begin{enumerate}
\item Pada tahapan rekayasa kebutuhan didapatkan 2 aktor utama, 2
  kebutuhan fungsional untuk sistem \emph{Lup Recoder} dan 10
  kebutuhan fungsional untuk sistem \emph{Lup Viewer}. Dua aktor
  utama yang terlibat di dalam sistem adalah dosen dan
  mahasiswa. Kebutuhan utama sistem adalah merekam berkas tugas,
  merekam aktivitas siswa, dan memiliki sifat \emph{platform
    agnostic}. Kedua sistem memiliki satu kebutuhan non-fungsional
  yaitu \emph{compatibility}. Kedua sistem harus \emph{compatible}
  dengan enam \emph{desktop environment} utama pada sistem operasi
  \emph{GNU/Linux}.

\item Pada tahapan perancangan didapatkan pemodelan \emph{class
    diagram} yang terdiri dari 6 \emph{class} pada sistem \emph{Lup
    Recorder} dan 12 \emph{class} pada sistem \emph{Lup Viewer}, serta
  tiga sampel pemodelan \emph{sequence diagram}. Hasil dari tahapan
  perancangan diimplementasikan dalam bentuk \emph{working code}
  dengan bahasa pemrograman \emph{Python}, dan antarmuka sistem
  diimplementasikan dengan bantuan \emph{widget toolkit} \emph{Qt}.

\item Terdapat lima pengujian yang dilakukan, yaitu pengujian unit,
  pengujian integrasi, pengujian validasi, \emph{automated testing}, dan pengujian
  \emph{compatibility}. Pengujian unit
  dilakukan dengan metode \emph{white-box testing} dan teknik
  \emph{basis path testing}, pengujian integrasi dilakukan dengan
  pendekatan \emph{use-based testing} menggunakan metode
  \emph{white-box testing} dan teknik \emph{basis path testing},
  pengujian validasi dilakukan dengan metode \emph{black-box testing}
  dengan pemilihan nilai \emph{input} menggunakan teknik \emph{boundary
    value analysis} dan \emph{equivalence partitioning}, \emph{automated
    testing} dilakukan dengan menggunakan bantuan kakas bantu \emph{Pytest} dan
  \emph{Gitlab-CI}, dan pengujian
  \emph{compatibility} dilakukan dengan menjalankan sistem pada enam
  \emph{desktop environment} pada sistem operasi \emph{GNU/Linux}.
  Pengujian unit dan
  integrasi yang dilakukan telah mencapai lebih dari 70\%
  \emph{code coverage} sehingga bisa dipastikan semua \emph{path}
  utama selesai diuji. Pengujian validasi dilakukan terhadap seluruh
  \emph{use case scenario} sehingga bisa dipastikan bahwa sistem yang
  dibangun memenuhi seluruh kebutuhan yang didefinisikan. \emph{Automated
    testing} dijalankan tanpa menghasilkan galat sehingga pengujian
  dapat dilakukan dengan otomatis dan memangkas waktu pengujian. Pengujian
  \emph{compatibility} telah memastikan sistem berjalan dengan baik
  pada enam \emph{desktop environment} berbeda.
\end{enumerate}

\section{Saran}

Saran untuk penelitian selanjutnya adalah sebagai berikut:

\begin{enumerate}
\item Aplikasi pencegah plagiarisme dengan menggunakan metode
  \emph{snapshotting} dan \emph{user activity logging} menyimpan semua
  data rekaman secara lokal. Maka terdapat kemungkinan adanya
  manipulasi data rekaman oleh mahasiswa yang memahami struktur data
  rekaman sistem. Saran untuk penelitian selanjutnya adalah dengan
  tidak menyimpan data secara lokal, melainkan dikirim dan disimpan
  langsung pada penyimpanan \emph{server}.

\item Sistem operasi yang didukung dapat ditambah dengan melakukan
  \emph{porting} modul perekam pada sistem ke sistem operasi lain
  seperti sistem operasi \emph{Microsoft Windows} dan \emph{Apple
    Macintosh}.
\end{enumerate}

%%% Local Variables:
%%% coding: utf-8
%%% mode: latex
%%% TeX-engine: xetex
%%% TeX-master: "skripsi"
%%% ispell-local-dictionary: "id"
%%% End:
