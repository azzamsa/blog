%%%%%%%%%%%%%%%%%%%%%%%%%%%%%%%%%%%%%%%%%%%%%%%%%%%%%%%%%%%%%%%%%%%%%%%
% BAB 1
%%%%%%%%%%%%%%%%%%%%%%%%%%%%%%%%%%%%%%%%%%%%%%%%%%%%%%%%%%%%%%%%%%%%%%%

\mychapter{1}{BAB 1 PENDAHULUAN}

% BAB I PENDAHULUAN
% BAB II PROFIL OBYEK PKL .
% BAB III TINJAUAN PUSTAKA
% BAB IV METODOLOGI
% BAB V HASIL DAN PEMBAHASAN
% BAB VI BAB VI PENUTUP

\section{Latar Belakang}
% 1. Objek penelitian
% a. mengapa testing ?

% obyek penelitian (O)
Saat ini perangkat lunak telah digunakan secara masif oleh
manusia. Semua bidang dalam setiap aspek kehidupan hampir seluruhnya
membutuhkan perangkat lunak. Mulai dari transportasi, kesehatan,
pendidikan, tenaga pembangkit listrik dan berbagai macam bidang-bidang
lainnya. Meningkatnya penggunaan perangkat lunak berpengaruh secara
langsung pada kebutuhan infrastruktur teknologi informasi. Mahal dan
sulitnya pemeliharaan infrastruktur teknologi informasi membuat
penyedia jasa infrastruktur tekologi informasi kian menjamur.
BiznetGio Nusantara merupakan salah satu diantaranya. BiznetGio
Nusantara menyediakan jasa penyewaan infrastruktur teknologi informasi
bernama \emph{NeoCloud}. \emph{NeoCloud} merupakan infrastruktur
\emph{cloud} berbasis \emph{OpenStack}. \emph{NeoCloud} melayani
puluhan perusahaan besar di Indonesia sehingga kegagalan sistem pada
\emph{NeoCloud} merupakan hal yang sangat fatal. Kegagalan perangkat
lunak dapat menyebabkan kerugian materi hingga kehilangan nyawa
\parencite{wong2009role}. Akan tetapi, kegagalan perangkat lunak dapat
diantisipasi dengan melakukan prosedur pengujian pada proses
pengembangan perangkat lunak. Pengujian dapat memprediksi kegagalan
lebih dini dan menghasilkan perangkat lunak yang berkualitas
\parencite{burnstein2006practical}.

Maka untuk mengantisipasi kegagalan dan menjadikan \emph{NeoCloud}
perangkat lunak yang berkualitas tinggi, dilakukanlah pengujian pada
\emph{NeoCloud}. Pada praktik kerja lapangan ini pengujian dilakukan
pada \emph{neo-cli}. \emph{Neo-cli} merupakan perangkat lunak bagian
dari \emph{NeoCloud} yang digunakan oleh pengguna \emph{NeoCloud}
untuk memudahkan pengaturan infrastruktur \emph{cloud} yang mereka miliki.
Pengujian yang akan dibangun pada
\emph{neo-cli} dilakukan pada beberapa tahapan, yaitu pada tahapan
\emph{unit} dan \emph{integration}. Pemilihan batas tahapan ini
ditentukan oleh pembimbing lapangan dengan beberapa
pertimbangan. Salah satunya adalah waktu praktik kerja lapangan yang
terbatas. Sedangkan metode pengujian yang digunakan adalah
\emph{white-box testing} dan \emph{black-box testing}. Metode
\emph{white-box testing} tidak dapat menemukan kebutuhan yang belum
diimplementasikan \parencite{dijkstra1970notes}. Hal ini dapat
diselesaikan dengan penggunaan metode \emph{black-box
testing}. Begitu juga dengan kekurangan metode \emph{black-box
testing} yang dapat menghasilkan \emph{test-case} yang tidak tepat dan
tidak dapat menemukan bagian yang belum diuji
\parencite{savenkov2008become}. Kelemahan pada metode
\emph{black-box testing} ini dapat diselesaikan dengan menggunakan
metode \emph{white-box testing}. Oleh karena itu, pada laporan ini
kedua metode tersebut digunakan. Pengujian semua \emph{path} atau
\emph{rigorous testing} pada proses \emph{white-box testing} tidak
mungkin dilakukan sehingga digunakan teknik \emph{basis path
testing} untuk menentukan \emph{path} yang akan diuji
\parencite{gregory2007path}. Begitu juga pada proses \emph{black-box
  testing}. Pengujian semua kemungkinan \emph{input} tidak dapat
dilakukan sehingga digunakan teknik \emph{equivalence partitioning} untuk
menentukan \emph{valid input} dan \emph{invalid input} yang akan
digunakan dan penggunaan teknik \emph{boundary value
  analysis} untuk memilih \emph{input} yang ada pada nilai-nilai
\emph{boundary}, karena nilai-nilai pada bagian tersebut memiliki
kemungkinan galat yang besar \parencite{presman2010software}. Proses
pengujian menyita banyak waktu \parencite{brooks1995mythical} sehingga
\emph{automated testing} juga digunakan pada laporan ini untuk
mempercepat proses pengujian.

Pada laporan ini pengujian pada \emph{neo-cli} akan dibangun pada
tahapan \emph{unit} dan \emph{integration} dengan menggunakan kedua
metode \emph{white-box testing} beserta \emph{black-box
  testing}. Pada proses \emph{white-box testing} akan digunakan teknik
\emph{basis path testing}. Sedangkan pada proses \emph{black-box
  testing} akan digunakan teknik \emph{equivalence partitioning} dan
\emph{boundary value analysis}. \emph{Automated testing} juga
digunakan pada pengembangan pengujian \emph{neo-cli} untuk mempercepat
proses pengujian.

\section{Rumusan Masalah}
\begin{enumerate}
\item Bagaimana perancangan dan implementasi pengujian dengan teknik
  \emph{basis path testing} pada perangkat lunak \emph{neo-cli} ?
\item Bagaimana perancangan dan implementasi pengujian dengan teknik
  \emph{equivalence partitioning} dan \emph{boundary value analysis}
  pada perangkat lunak \emph{neo-cli} ?
\end{enumerate}

\section{Tujuan}

\begin{enumerate}
\item Merancang dan mengimplementasikan pengujian dengan teknik
  \emph{basis path testing} pada perangkat lunak \emph{neo-cli}.
\item Merancang dan mengimplementasikan pengujian dengan teknik
  \emph{equivalence partitioning} dan \emph{boundary value analysis}
  pada perangangkat lunak \emph{neo-cli}.
\end{enumerate}


\section{Manfaat}

Pembangunan pengujian pada perangkat lunak \emph{neo-cli}
diharapkan dapat memberikan kemudahan bagi pengembang untuk menemukan
\emph{bug}. Baik \emph{bug} yang tersebunyi maupun yang muncul karena
adanya penambahan fitur. Selain itu, laporan ini diharapkan memberikan
informasi yang menjelaskan tentang proses pengujian
dalam perangkat lunak. Baik dengan pendekatan \emph{black-box} atau
pun \emph{black-box} dan teknik-teknik yang digunakan di dalamnya.

\section{Batasan Masalah}

\begin{enumerate}
\item Beberapa modul tidak memiliki dukungan untuk proses pengujian. Seperti
  modul \emph{ncurses}
\item \emph{Unit} yang berinteraksi dengan dunia luar tidak di uji
  secara \emph{end-to-end} pada proses \emph{integration testing}.
\end{enumerate}


\section{Sistematika Pembahasan}
\noindent
\textbf{BAB I : PENDAHULUAN}

Bab ini menjelaskan tentang latar belakang masalah, rumusan masalah,
tujuan, manfaat, batasan masalah, dan sistematika pembahasan.\par\null\par

\noindent
\textbf{BAB II : PROFIL OBYEK PKL}

Bab ini menjelaskan tentang sejarah perusahaan, visi dan misi
perusahaan serta struktur organisasi perusahaan. \par\null\par

\noindent
\textbf{BAB III : TINJAUAN PUSTAKA}

Bab ini menjelaskan dasar teori-teori yang digunakan untuk melakukan
pengujian pada suatu perangkat lunak. Di dalamnya juga terdapat
penjelasan teknologi yang digunakan untuk membangun
pengujian pada perangkat lunak. \par\null\par

\noindent
\textbf{BAB IV : METODOLOGI}

Bab ini menjelaskan metode-metode yang digunakan selama pengujian. Di
dalamnya terdapat alur seperti proses analisis kebutuhan pengujian,
perancangan dan implementasi kasus uji, dan penarikan kesimpulan.
\par\null\par

\noindent
\textbf{BAB V : HASIL DAN PEMBAHASAN}

Bab ini menjelaskan hasil dan pembahasan pengujian. Hasil dan
pembahasan tersebut mencangkup langkah-langkah seperti analisis
kebutuhan pengujian, dan perencanaan dan implementasi kasus uji \par\null\par

\newpage

\noindent
\textbf{BAB VI : PENUTUP}

Bab ini menjelaskan kesimpulan selama proses pengujian, saran untuk pengujian
selanjutnya dan penjelasan terkait keberlanjutan pengujian yang dibangun.


%%% Local Variables:
%%% mode: latex
%%% TeX-master: "pkl"
%%% End:
