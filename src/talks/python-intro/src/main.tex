\input{preamble.tex}
\input{title.tex}
\newfontfamily\DejaSans{DejaVu Sans}

\begin{document}

\maketitle

\section{Yak Shaving}

\subsection{Install}

\begin{xframe}{Installing Python on Windows}
  % You need a \hl{style/images} directory
  \begin{center}
    {\huge Get the installer from} \\
    \href{https://www.python.org/downloads/windows/}{python.org/downloads/windows/}
  \end{center}
\end{xframe}

\begin{xframe}{Installing Python on GNU/Linux}
  \begin{center}
    \huge It's already installed {\DejaSans 😎}
  \end{center}
\end{xframe}

\subsection{Setup}
\label{subsec:label}

\begin{xframe}{Setup}
  \begin{center}
    \huge SETUP
  \end{center}
\end{xframe}


\section{Basic}

\subsection{Printing}

\begin{xframe}{Printing}

  % \begin{minted}[mathescape,
  % linenos,
  % numbersep=5pt,
  % gobble=2,
  % frame=lines,
  % framesep=2mm]{python}
  % print ''Hallo Pondok Mahasiswa`'
  % \end{minted}


\begin{minted}[linenos,frame=lines]{python}
print "Hallo, Pondok Mahasiswa"
\end{minted}
\end{xframe}

\begin{xframe}{Comment}
\begin{minted}[linenos,frame=lines]{python}
print "Hallo Pondok Mahasiswa"
# I am comment :)
\end{minted}
\end{xframe}

\begin{xframe}{Math}
\begin{minted}[linenos,frame=lines]{python}
print 10 + 2 - 3 / 6 * 7
# more %, <, >, <=, >=
\end{minted}
\end{xframe}

\begin{xframe}{Variables \& More Printing}
\begin{minted}[linenos,frame=lines,breaklines]{python}
nama = "Budi"
umur = 10
tinggi_badan = 30
berat_badan = 50.2

print "umur saya :", umur
print "nama saya : %s" % nama
print "saya %s, umur saya %d tahun, berat badan saya %0.2f" % (nama, umur, berat_badan)
\end{minted}
\end{xframe}

\begin{xframe}{Function}
\begin{minted}[linenos,frame=lines,breaklines]{python}
def penjumlahan(x, y):
    print "hasil penjumlahan : %d" % (x + y)

penjumlahan(2, 3)
\end{minted}
\end{xframe}

\begin{xframe}{Function \& Return Value}
\begin{minted}[linenos,frame=lines,breaklines]{python}
def penjumlahan(x, y):
    return x + y

def perkalian(x, y):
    return x * y

hasil_jumlah = penjumlahan(2, 3)
hasil_kali = perkalian(10, 2)

print "Hasil penjumlahan : %d dan Hasil perkalian : %d " % (hasil_jumlah, hasil_kali)
\end{minted}
\end{xframe}


\begin{xframe}{Prompting}
\begin{minted}[linenos,frame=lines,breaklines]{python}
nama = raw_input("siapa nama anda ?")
umur = int(raw_input("berapa umur anda ?"))

print "Hai, %s. Anda berumur %d tahun" % (nama, umur)
# what is int(), str(), .. ?
\end{minted}
\end{xframe}

\begin{xframe}{Make Decision}
\begin{minted}[linenos,frame=lines,breaklines]{python}
def hitung_jarak(km):
    if km > 50:
        print "sangat jauh"
    elif km == 50:
        print "cukup jauh"
    elif km == 30:
        print "sedang"
    else:
        print "dekat"

hitung_jarak(10)
\end{minted}
\end{xframe}

\begin{xframe}{Loop \& List}
\begin{minted}[linenos,frame=lines,breaklines]{python}
santri = ["ahmad", "imam", "umar", "ali"]

for anak in santri:
    print "nama santri %s" % anak

# what about while-loop ?
\end{minted}
\end{xframe}

\begin{xframe}{Dictionary}
\begin{minted}[linenos,frame=lines,breaklines]{python}
data_santri = {'nama': 'ahmad', 'umur': 40, 'tinggi_badan': 10+10}

print data_santri['nama']
print data_santri['umur']
print data_santri['tinggi_badan']
\end{minted}
\end{xframe}

\begin{xframe}
\begin{center}
  {\huge It's just \emph{gentle} introduction} \\
  keep learning.
  \end{center}
\end{xframe}

\section{Question ?}

\end{document}
