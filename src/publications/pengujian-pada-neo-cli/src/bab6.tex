
%%%%%%%%%%%%%%%%%%%%%%%%%%%%%%%%%%%%%%%%%%%%%%%%%%%%%%%%%%%%%%%%%%%%%%%
% BAB 6
%%%%%%%%%%%%%%%%%%%%%%%%%%%%%%%%%%%%%%%%%%%%%%%%%%%%%%%%%%%%%%%%%%%%%%%

\mychapter{6}{BAB 6 PENUTUP}

\section{Kesimpulan}

Berdasarkan proses pembangunan pengujian yang di dalamnya
terdapat tahapan seperti analisis kebutuhan pengujian, dan
perancangan dan implementasi kasus uji. Dapat disimpulkan bahwa:

\begin{enumerate}
\item Nilai cakupan pengujian tidak dipatok dengan nilai umum
  80\%-85\%. Hal ini disebabkan karena nilai tersebut tidak akan
  mungkin dicapai dengan keberadaan modul-modul yang tidak mendukung
  proses pengujian.
\item Beberapa modul dengan \emph{cyclomatic complexity} yang tinggi
  menyulitkan proses pengujian. Sudah selayaknya dilakukan
  \emph{refactoring} pada modul-modul tersebut.
\item Pengujian harus dilakukan secara terisolasi. Oleh karena itu,
  dilakukan \emph{mock} pada bagian lain di luar sesi pengujian
  tersebut.
\item \emph{Test script} untuk \emph{automated testing} dapat
  melakukan pengujian secara otomatis. Hal ini dapat mempercepat proses
  pengujian.
\item Penggunaan teknik \emph{basis path testing} memberikan batas
  \emph{path} yang akan diuji. \emph{Path} yang terbatas memberikan
  tolak ukur selesainya pengujian. Berbeda dengan \emph{rigorous
    testing} yang berusaha menguji banyak \emph{path} dan tidak
  memiliki tolak ukur jelas tentang ukuran selesainya sebuah
  pengujian.
\item Penggunaan teknik \emph{equivalence partitioning} dan
  \emph{boundary value analysis} memberikan batasan masukan untuk
  pengujian. Dengan teknik tersebut pengujian dapat dilakukan dengan
  data masukan yang sangkil (efisien). Tidak dengan data masukan yang sembarang.
\end{enumerate}

\section{Saran}

Banyak modul yang memiliki nilai \emph{cyclomatic complexity} yang
tinggi sehingga membutuhkan \emph{refactoring}. Modul-modul lain yang
berhubungan dengan dunia luar (\emph{outside world}) selayaknya
diproses dengan \emph{mock} dan tidak menggunakan \emph{resource} yang
sebenarnya. Penggunaan \emph{resource} asli (tanpa \emph{mock}) hanya
akan menyulitkan proses pengujian. Oleh karena itu, diharapkan agar
semua modul yang berhubungan dengan dunia luar untuk diuji menggunakan
\emph{mock}.

\section{Keberlanjutan}

Proses pengujian ini akan dilanjutkan dengan membuat \emph{mock} pada
modul-modul lain yang berhubungan dengan dunia luar (\emph{outside
  world}). \emph{Refactoring} juga akan dilakukan pada modul-modul
dengan \emph{cyclomatic complexity} yang tinggi. \emph{Automated
  testing} nantinya hanya akan dilakukan pada pengujian \emph{unit}
dan \emph{integration}. Hal-hal yang memudahkan proses pengujian juga
akan ditambahkan, seperti proses \emph{code review} menggunakan
\emph{gerrit}. Tata pelaksanaan \emph{pull request} akan diketatkan
pada bagian \emph{code style guide}, dalam hal ini yaitu penggunaan
\emph{PEP8}.


%%% Local Variables:
%%% mode: latex
%%% TeX-master: "pkl"
%%% End:
